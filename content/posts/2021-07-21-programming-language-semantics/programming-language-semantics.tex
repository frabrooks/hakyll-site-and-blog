---
title: Programming Language Semantics via Hoare Logic

desc: In 1967, as an alternative to <span class="math inline">\(\lAngle\)</span> operational <span class="math inline">\(\rAngle\)</span> semantics, Floyd produced his seminal paper ‘Assigning Meanings to Programs’ in which a program is given semantics via attachment of propositions to the connections in a flow chart with nodes as commands.

technical: yes

tags:
 - Hoare Logic
 - Agda
 - Program Correctness
 - Verification
 - Formal Proof

---
% DOCUMENT CLASS
\documentclass[oneside,12pt]{article}

\usepackage{hyperref}

\usepackage{graphicx}
\usepackage[dvipsnames]{xcolor}
\usepackage{tikz}
\usepackage{verbatim}

\usepackage{amsmath}
\usepackage{mathtools}

\usepackage{agda}

% \input{font-business}

% Caption options
\usepackage[font=footnotesize,labelfont=bf]{caption}

\usepackage{lipsum}

% Bold face small caps
% \usepackage{bold-extra}

%\input{unicode}
%\usepackage{eufrak}
%\usepackage{mathabx}
% Use Chancery Font


% Includes "References" in the table of contents
\usepackage[nottoc]{tocbibind}


\usepackage{epigraph}
\usepackage{varwidth}

\usepackage{ stmaryrd }% For \llbracket \rrbracket
% Used in \tctrip def


% For NB ligature
\newcommand\NB[1][0.1]{NB \; \; } % default kern amount: -0.3em

% Style for Agda snippet math script replacement
% (emphasised agda definitions)
\newcommand{\agdamath}[1]{\emph{\texttt{\!#1}}}

% style for Mini-Imp construct/mechanism
\newcommand{\impcode}[1]{\texttt{#1}}

\newcommand{\codevar}[1]{\ensuremath{#1}}

\newcommand{\textM}[1]{$#1$}

% For constants in agda-snippets
% Place input in circle (arg should be one numeral 1-9)
\newcommand{\constv}[1]{\raisebox{.5pt}{\textcircled{\raisebox{-.9pt} {#1}}}}

% Hoare's original notation for partial correctness
% \newcommand{\hpc}[3]{\textM{#1 } \{\!\!\{\  \textM{ #2 }  \}\!\!\} \textM{ #3}}
\newcommand{\hpc}[3]{\! \textM{#1} \;  \{\!\!\{ \; \textM{#2} \;  \}\!\!\}  \; \textM{#3} \!}
% Inline version with tighter spacing
\newcommand{\hpcil}[3]{ \!\!\!  \textM{#1} \!\!\!  \{\!\!\{  \textM{#2}  \}\!\!\}  \!\!\! \textM{#3} \!\!\!  }

% Gries then uses this for total correctness, but confusingly in many
% expositions these days it is used to denote partial correctness!
\newcommand{\gtc}[3]{\!  \{\!\!\{ \; \textM{#1} \; \}\!\!\} \; \textM{#2} \;  \{\!\!\{ \; \textM{#3} \; \}\!\!\} \!}
% Inline version with tighter spacing
\newcommand{\gtcil}[3]{ \!\!\!\!   \{\!\!\{  \!\!\!  \textM{#1} \!\!\! \}\!\!\} \textM{#2}   \{\!\!\{  \!\!\! \textM{#3} \!\!\!  \}\!\!\} \!\!\!  }

% This reports/project's notation
% Partial Correctness Hoare-Triple
\newcommand{\pctrip}[3]{$\large{\lAngle}\normalsize{#1}\large{\rAngle}%
  \normalsize{#2}\large{\lAngle}\normalsize{#3}\large{\rAngle}\normalsize$}

% Total Correctness Hoare-Triple
\newcommand{\tctrip}[3]{$\large{\llbracket}\normalsize{#1}\large{\rrbracket}\,%
  \normalsize{#2}\,\large{\llbracket}\normalsize{#3}\large{\rrbracket}\normalsize$}


\newcommand{\wpre}[2]{$\textit{wp}(#1,#2)$}

\newcommand{\wlpre}[2]{$\textit{wlp}(#1,#2)$}

% Agda code state transformer symbol
\newcommand{\stateT}{\ensuremath{S\Delta}}

% def above equal sign
\newcommand{\eqdef}{$\stackrel{\text{\tiny def}}{=}$}

% filled circle (empty state) symbol
\newcommand{\circfill}{\tikz\draw[black , fill=black] (0,0) circle (.6ex);}



\begin{document}


\subsection{Axiomatic Semantics via Hoare Logic}


In 1967, as an alternative to $\lAngle$ operational $\rAngle$ semantics, Floyd\cite{Floyd1967Flowcharts} produced his seminal paper `Assigning Meanings to Programs' in which a program is given semantics via attachment of propositions to the connections in a flow chart with nodes as commands. In Floyd's deductive system, whenever a command (a node) is reached via a connection whose associated proposition is true, then, if execution of the program leaves that node, it will leave through a connection whose associated proposition is also true.

\begin{quote}

``A \emph{semantic definition} of a particular set of command [program] types, then, is a rule for constructing, for any command \textM{c} of one of these types, a \emph{verification condition} \textM{V_c(P;Q)} on the antecedents and consequents of \textM{c}''. - \footnotesize Floyd\cite{Floyd1967Flowcharts}

\end{quote}

The principle idea is that rather than define a program (however large or small) by the way it should be executed, a program can be defined by the antecedents upon the state space that must be true before execution --- hereafter referred to as \emph{preconditions} --- and the associated consequents upon the state space that can be guaranteed to be true after execution --- hereafter referred to as \emph{postconditions} --- thus freeing the semantics from concerns of the \emph{how} in favour of the \emph{what.}


These `antecedents/consequents upon the state space' are first-order logic predicates or propositions and the state space is taken most generally to be a set of pairs of identifiers and values; again the formulation here shields us from implementation details such as whether these `identifiers' identify memory addresses within a machine or Post-it Notes on a wall.

Later then, in 1969, Hoare\cite{hoare1969axiomatic} built upon and expanded Floyd's work\footnote{Thus Hoare logic is sometimes referred to as Floyd-Hoare Logic}, applying the system to text (\; \pctrip{P}{S}{Q} \;) rather than to flow charts, creating a \emph{deductive system} for reasoning about the correctness of programs ( \tctrip{p}{S}{Q} ) as we would more naturally recognise them. Central to Hoare's system is the notion of a \emph{Hoare triple} which is a reformulation of Floyd's verification condition \textM{V_c(P;Q)}. A Hoare triple associates a precondition, or a state, before execution of a particular program with a resultant postcondition, or state, after execution.\footnote{\NB Here, as in much of the literature, preconditions and postconditions and the actual subsets of the state space that they describe are used interchangeably. i.e. \impcode{False} = \emptyset and \impcode{True} = \textM{S} where \textM{S} is the whole state space.} 



\end{document}


